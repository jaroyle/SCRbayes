\documentclass[12pt]{article}
\usepackage{amsmath}

\usepackage{amssymb}    % used for symbols in figure legends
%\usepackage{natbib}    % used in formatting of citations and bibliography
\usepackage{url}
\usepackage{framed}
\usepackage{float}
\input{psfig}
%\input{boldGreek}

%\linenumbers
%\raggedright
%\setlength{\parindent}{.25in}


%\lhead{}
%
%\rhead{\thepage}
%\renewcommand\headrulewidth{0.4pt}

\input{boldGreek}

\floatstyle{plain}
\floatname{panel}{Panel}
\newfloat{panel}{h}{txt}

\renewcommand{\baselinestretch}{1.0}
\setlength{\textwidth}{6.5in}
%\setlength{\evensidemargin}{0.1875in}
%\setlength{\oddsidemargin}{0.1875in}
\setlength{\evensidemargin}{0in}
\setlength{\oddsidemargin}{0in}

\setlength{\textheight}{8.425in}
%\setlength{\headheight}{.5in}
%\setlength{\headsep}{.5in}
%\setlength{\parindent}{.25in}
\setlength{\topmargin}{0.375in}



% display page layout
%\usepackage{layouts}
%\newcommand\showpage{%
%    \setlayoutscale{0.27}\setlabelfont{\tiny}%
%    \printheadingsfalse\printparametersfalse
%    \currentpage\pagedesign}

% paragraph formatting
\usepackage{indentfirst}  % indent first line of paragraph in new sections
%\usepackage{setspace}
%\singlespacing
%

\usepackage{graphicx}

\floatname{panel}{Panel}
\newfloat{panel}{h}{txt}


%\usepackage[nomarkers,tablesfirst,nolists]{endfloat}   % places all tables and figures at end of document





\begin{document}

\begin{center}
\Large \textbf{
SCRbayes: MCMC for a Class of Spatial Capture-Recapture Models

}
\end{center}
%

\noindent 

\noindent


\section{Overview}

We are considering a capture-recapture model in which the observations
$y(i,j,k)$ are binary encounters of individual $i$, in trap $j$ and
for period (e.g., night) $k$.
The basic model is:
\[
 y(i,j,k) \sim \mbox{Bern}(\pi(i,j,k))
\] 
with
\[
 \mbox{cloglog}(\pi(i,j,k)) = \alpha_{0} + \alpha_{1}x_{ijk} - (1/\sigma)*d_{ij}^{2}
\]

There could be any number of fixed covariates in this model but here
we have $x_{ijk}$ which I assume to be an indicator of previous capture.
i.e., a behavioral response. 
In this case, the behavioral response operates at the trap level. 

To attack this problem by MCMC we have to deal with 3 basic elements:
\begin{itemize}
\item[(1)] We don't know $N$. For that we use data augmentation. 
\item[(2)] We have to update the ``structural parameters'' of the model.
These are the regression coefficients which include $\alpha_{0}$,
$\alpha_{1}$ and $\sigma$.  I will consider this part of the algorithm
first since I think it can be economized considerably with careful thought
and programming.
\item[(3)] We have to update the spatial effects ${\bf s}$ which are
the individual home range centers. These are contained in the $d_{ij}$
terms above. 
\end{itemize}


\section{Dealing with unknown $N$}

This is dealt with by data augmentation (Royle et al. 2007). The idea
is to introduce a large number of all-zero encounter histories and
recognize that the augmented data set (containing the all-zeros) is a
zero-inflated version of the known-N model.  To implement that in
WinBUGS we introduce a collection of binary indicator variables
$z_{i}$ for each of the $i=1,2,\ldots, M$ individuals ($M$ = size of
augmented data set), and the latent variables $z_{i}$ are updated as
part of the MCMC algorithm.  Thus, when $z_{i}=1$, then the all-zero
observation ${\bf y}_{i}$ is an element of the ``real'' population.
That is, ${\bf y}_{i} = {\bf 0}$ is a sampling zero as opposed to a
fixed zero.

I will outline how this is done in an update of the document.  The key
thing to keep in mind is that the rest of the MCMC algorithm proceeds
{\bf conditional on the present values of $z_{1},z_{2},\ldots,z_{M}$}.
This really simplifies things.


\section{Updating the Structural Parameters}

To do this, \mbox{\tt SCRbayes} 
uses a standard component-wise Metropolis-Hastings
algorithm where we simply compute the log-likelihood for the
observations twice, once given the current parameter value and a 2nd
time given a candidate parameter value. We use the random walk proposal
distribution exclusively.  Let ${\bf y}$ be {\bf all} the observations
{\bf for which $z=1$} stacked up on each other in a vector format
($z=1$ means that these are all of the individuals in the actual
population).  Let ${\bm \alpha}$ be the vector of regression
parameters. Somehow we obtain a candidate value of the parameter, most
likely by sampling its components from a normal distribution having
mean ${\bm \theta}$ (this is a random walk Metropolis sampler).  If we
use that method then define:
\[
r = exp( logL({\bf y}|{\utheta}^{*}) - logL({\bf y}|{\utheta}))
\]
and we accept ${\utheta}^{*}$ with probability $r$. We do this in
practice by flipping a coin with probability $r$....generating a
uniform random number... and then if that number is $< r$ we accept.
In practice we might also do each {\it component} of ${\bf \utheta}$
individually so as to improve mixing of the Markov chains. In the
current version of the R code, \mbox{\tt SCRe.fn}, I am updating
$\sigma$ and $\alpha_{0}$ {\it together} and then doing $\alpha_{1}$ by
itself ({\bf NOTE:} sorry but my notation here is not entirely
consistent with the R code).


The key issue in implement MCMC for this model is {\it efficient
  computation of the log-likelihood}.  For a situation with 100 traps
and 500 individuals (including pseudo-individuals) and 12 trap
occasions there are 600,000 total observations. This is an enormous
data set to work with.  We have to subset the 500 individuals to
include only those for which $z=1$ and then compute the
log-likelihood.  We can reduce the size considerably by summing over
the $K=12$ replicates which is allowable if there are no ``time
effects''. However, we would like to compute $logL$ efficiently for
the general case and so we focus on that for now.

Ok, so, what does the $logL$ look like? It is just Bernoulli of the
form:
\[
L = \prod_{i,j,k}  \pi(i,j,k)^{y(i,j,k)} (1-\pi(i,j,k))^{1-y(i,j,k)}
\]
where
\[
 \pi(i,j,k) = 1-exp(- 
exp( \alpha_{0} + \alpha_{1}x_{ik} - (1/\sigma)*d_{ij}^{2} ))
\]

It is worth noting that if $y=0$ then the first part of $L$ is {\it
  always} 1.0 whereas if $y=1$ then the 2nd part of $L$ is always 1.0.
Thus we can express the likelihood L as:
\[
L = \prod_{y=1}  \pi(i,j,k) \prod_{y=0} (1-\pi(i,j,k))
\]
Further, in terms of  the log-likelihood we have
\[
logL = \sum_{y=1} log(  \pi(i,j,k) )  \prod_{y=0} 
- exp( \alpha_{0} + \alpha_{1}x_{ik} - (1/\sigma)*d_{ij}^{2} )
\]
While my notation is not explicit here I want to emphasize that the
2nd term includes the summed values of $1-\pi(i,j,k)$ only for those
values of $y$ that are equal to 0.

We should be able to come up with some very efficient ways of
evaluating $logL$ either explicitly or by using some tricks, such as
only computing certain required components, or saving bits from
previous evaluations, etc..  We will have to work on this.

Ok, so consider a MH algorithm for the 3 parameters. If we update each
of those individually the MH algorithm is structured like this:

\begin{itemize}
\item (A1) Evaluate $logL({\bf y}|\alpha_{0},\alpha_{1},\sigma)$

\item (A2)  Draw a candidate value of $\alpha_{0}$, say $\alpha_{0}^{*}$

\item (A3) Evaluate $logL({\bf y}|\alpha_{0}^{*},\alpha_{1},\sigma)$

\item (A4) Accept with probability $r$ according to the rule given above.
\end{itemize}

\begin{itemize}
\item (B1) Evaluate $logL({\bf y}|\alpha_{0},\alpha_{1},\sigma)$. Obviously
 this should not require a complete reevaluation as the calculations
 have been done in the A step.

\item (B2)  Draw a candidate value of $\alpha_{1}$, say $\alpha_{1}^{*}$

\item (B3) Evaluate $logL({\bf y}|\alpha_{0},\alpha_{1}^{*},\sigma)$

\item (B4) Accept with probability $r$ according to the rule given above.
\end{itemize}

\begin{itemize}
\item (C1) Evaluate $logL({\bf y}|\alpha_{0},\alpha_{1},\sigma)$. Obviously
 this should not require a complete reevaluation as the calculations
 have been done in the A step.

\item (C2)  Draw a candidate value of $\sigma$, say $\sigma^{*}$

\item (C3) Evaluate $logL({\bf y}|\alpha_{0},\alpha_{1},\sigma^{*})$

\item (C4) Accept with probability $r$ according to the rule given above.
\end{itemize}



\subsection{Alternative distance functions}

Easy to do. For now we only allow for estimating the exponent $\theta$
on distance so that it can range from $\theta = .5$ (exponential) to
$\theta = 1$ (Gaussian).  


\section{The Data Augmentation Part of the Algorithm}

{\it 
This step of the MCMC algorithm determines which of the data
augmentation variables, $z$, are equal to 1. It is the $z=1$ subset of
individuals for which the calculations in the previous Section are
carried out.
}


Bayesian analysis of this class of models is facilitated by data
augmentation.  One of the motives for doing this is that Bayesian
analysis is carried out conditional on the individual activity centers
${\bf s}_{i}$. In principle, then, we could specify fairly complex
models describing how the activity centers are distributed in space
and then, in the MCMC, we only have to be able to simulate the
resulting point process appropriately.  Conversely, if we tried to do
integrated likelihood then we would have an $N-$fold integral to carry
out.

Operationally, using data augmentation, we add a large number of
all-zero encounter histories to our data set. If $n$ is the number of
individuals encountered then let $M-n$ be the number we add to the
data set, chosen so that $M$ is a reasonable upper bound on the actual
population size $N$. Note, in practice, we probably choose $M$ by
trial and error. The considerations are this: We want $M$ to be as
small as possible in order to save calculations but if it is too
small, then the posterior of $N$ will be truncated. That is bad.  So
we have to fiddle around with that until we get it about right.

Given the augmented data set, the model contains a set of binary
indicator variables $z_{i}$ for $i=1,2,\ldots,M$.  These are
``zero-inflation'' variables such that if $z_{i}=1$ then individual
$i$ is a member of the population that was exposed to sampling. In
other words, the observations ${\bf y}_{i}$, correspond to data that
arose by the binomial sampling process. Conversely, if $z_{i}=0$, then
the observations ${\bf y}_{i}$ are {\it fixed} zeros. In this way,
i.e., by zero-inflation, data augmentation accounts for the fact that
our augmented data set contains too many zeros.

The model for the $z_{i}$ variables is $z_{i} \sim
\mbox{Bernoulli}(\psi)$.  As part of the MCMC algorithm we have to
impute the ``missing'' $z_{i}$ values. Note that $z_{i}=1$ for the
$i=1,2,\ldots,n$ observed individuals, necessarily.  Thus we only need
to update $z_{i}$ for the augmented pseudo-individuals. This is a
simple applications of Bayes' rule:
\[
 Pr(z=1|{\bf y} = 0) = \frac{Pr({\bf y}=0|z=1)Pr(z=1) }{Pr({\bf y}=0)}
\]
Recall
\[
 \pi(i,j,k) = 1-exp(- 
exp( \beta_{0} + \beta_{1}x_{ik} - (1/\sigma)*d_{ij}^{2} ))
\]
thus
\begin{eqnarray}
Pr({\bf y}_{i} = 0|z_{i}=1) &=&
 \prod_{j} \prod_{k} exp(- exp( \beta_{0} + \beta_{1}x_{ik} -  (1/\sigma)*d_{ij}^{2} )) \\
 & = & exp(- \sum_{j} \sum_{k} exp( \beta_{0} + \beta_{1}x_{ik} -  (1/\sigma)*d_{ij}^{2} ))
\end{eqnarray}
which, clearly, some economies should be realizable.  In particular,
we should be able to compute this by recycling calculations from the
previous step although the current version of the MCMC algorithm does
not make an effort to do that.  As time permits or we have a post-doc
or something then we can work on that.  To get the denominator (the
marginal probability $Pr({\bf y} = 0)$) of this Bayes' rule quantity,
note that $Pr({\bf y}=0|z=0) = 1$.

Once these calculations are done then we simply flip coins for
$i=n+1,\ldots,M$ having probabilities $Pr(z=1|{\bf y} = 0)$ and those
coin flips determine the current values of $z_{n+1},\ldots,z_{M}$.



\section{Updating $\{ {\bf s}_{i} \}$ -- the Activity Centers}

Right now we consider only the case where ${\bf s}$ is uniform on ${\cal
  S}$, the collection of points defining the state-space.  If $z=0$
then we draw ${\bf s}$ from the prior and it is accepted with
probability 1\footnote{actually we draw it randomly from a
  neighborhood and accept it with probability 1 as long as its not
  close to the edge, in which case an adjustment is made for asymmetry
  of the proposal}.  If $z=1$ then we have
to look at the data to decide if a candidate value is acceptable or not.

The strategy is to use a ``random walk'' proposal distribution for
each $s$.
Candidates are drawn by perturbing the current value locally. Specifically, 
we start off with a collection of neighbors for point s, say ${\cal N}(s)$ which are 
enumerated before the MCMC iterations are begun.  The size of this set can be arbtirary.
Smaller neighborhoods require fewer calculations but mix more slowly.

So we draw our candidate from that set. Then the actual full conditionals are evaluated.....
L(y|s^{*}) 
and compare to L(y|s)

Note for a small enough state-space grid then we could draw directly
from the multinomial full conditional distribution but I think
the calculation is too expensive for a fine state-space grid. 




3 cases require attention

(1) The $n$ observed individuals

(2) The $n_{0}$ individuals for which $z=1$ but $y=0$. They all have
the same full-conditional distribution

(3) Individuals for which $z=0$ are drawn from the prior distribution.






\subsection{Covariates on Density}

We can allow easily for covariates to influence ${\bf s}$. 
In particular, under the discrete state-space formulation, in that case then
${\bf s}_{i}$ is a multinomial trial.........

In the context of data augmentation there is a slight blip. The intercept is not identifiable.
Because that information is contained in $\psi$ See Royle et al. (in review)......


When we have a covariate that affects density, we just modify the full
conditional by inserting the multinomial prior distribution:
\[
 \pi({\bf s}) = \frac{ exp( \beta z({\bf s}) ) }
 \frac{ \sum_{s} exp( \beta z({\bf s}) ) }
\]
In this case, we have to update the parameter $\beta$ which we do
using a Metropolis-Hastings step with a random-walk proposal.








\section{Goodness-of-Fit using a Bayesian p-value}

The current version of the R code (SRCe.fn) has a GoF evaluation built
into it.






\end{document}

